\documentclass[11pt]{amsart}

%%%%%%%%%%%%%%%%%%%%%%%%%%%%%%%%%%%%%%%%%%%%%%%%%%%%
\usepackage{latexsym}
%\usepackage{graphics}
\usepackage{amssymb}
\usepackage{amsmath}
\usepackage{amsthm}

\usepackage{geometry}
\geometry{letterpaper}
%\usepackage{tikz}
%%%%%%%%%%%%%%%%%%%%%%%%%%%%%%%%%%%%%%%%%%%%%%%%%%%%
\newtheorem{theorem}{Theorem}
\newtheorem{definition}[theorem]{Definition}
\newtheorem{proposition}[theorem]{Proposition}
\newtheorem{corollary}[theorem]{Corollary}
\theoremstyle{remark}
\newtheorem{remark}[theorem]{Remark}
\newtheorem{exercise}{Exercise}
%%%%%%%%%%%%%%%%%%%%%%%%%%%%%%%%%%%%%%%%%%%%%%%%%%%%
\newcommand{\NN}{\mathbb{N}}
\newcommand{\ZZ}{\mathbb{Z}}
\newcommand{\QQ}{\mathbb{Q}}
\newcommand{\RR}{\mathbb{R}}
\newcommand{\CC}{\mathbb{C}}

\title{\LaTeX~sample}
\author{Gottfried Leibniz}
\date{November 14, 1716}

\begin{document}
\maketitle

\begin{theorem}
The derivative of a function $f$ at $c$ is given by
\[
  f'(c) = \lim_{x\to c} \frac{f(x)-f(c)}{x-c}
\]
provided this limit exists.
\end{theorem}
\begin{proof}
By definition, we know that
\begin{equation}\label{eqn:DerivDef}
  f'(c) = \lim_{\Delta x\to 0} \frac{f(c+\Delta x)-f(c)}{\Delta x}.
\end{equation}
Let $x = c+\Delta x$.  Then $x\to c$ as $\Delta x\to 0$.  So, replacing
$c+\Delta x$ by $x$ and $\Delta x$ by  $x-c$ in (\ref{eqn:DerivDef}),
we find that
\[
  f'(c) = \lim_{x\to c}\frac{f(x)-f(c)}{x-c}
\]
as desired.
\end{proof}

\end{document}